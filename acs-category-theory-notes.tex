\documentclass[a4paper, 12pt]{article}
\usepackage{a4wide}

\title{L108: Category Theory, Type Theory, and Logic\\
  ACS Category Theory Notes 2017}
\usepackage{nth}
\date{\nth{5} October 2017}
\author{Marcelo Fiore\\ Dhruv Makwana\\ Shaun Steenkamp}

\usepackage{fourier} % Better font
\usepackage{inconsolata} % mono spaced font
\usepackage{amssymb,amsthm,amsmath}
\usepackage{fdsymbol}
\usepackage{framed}
\usepackage{tikz}
\usepackage{tikz-cd}
\usepackage{microtype}

\theoremstyle{definition}
\newtheorem{definition}{Definition}
\newtheorem{lemma}{Lemma}
\newtheorem{theorem}{Theorem}
\newtheorem{exercise}{Exercise}

%%%% Stylistic
\setlength\parindent{0pt}
\setlength\parskip{0.5em}

%%%% Category Theory Macros
\newcommand {\cat}{%
    \mathbf%
}
\newcommand {\domain}[1] {%
    \mathrm{dom}(#1)%
}
\newcommand {\codomain}[1] {%
    \mathrm{cod}(#1)%
}
\newcommand {\idarrow}[1][] {%
    \mathbf{1}{#1}%
}
%%%% Macros for specific categories
\newcommand {\Cat}{%
    \cat {Cat}%
}
\newcommand {\Mon}{%
    \cat {Mon}%
}
\newcommand {\Poset}{%
    \cat {Poset}%
}
\newcommand {\Rel}{%
    \cat {Rel}%
}
\newcommand {\Sets}{%
    \cat {Sets}%
}
\newcommand {\Groups}{%
    \cat {Groups}%
}
\newcommand {\Graphs}{%
    \cat {Graphs}%
}

\begin{document}

\maketitle

\tableofcontents

\newpage
\section{Lecture 1 (\nth{5} October): How to think Categorically}
\vspace*{-5mm}\hspace*{7.75mm}
Marcelo Fiore, Dhruv Makwana, Shaun Steenkamp
\bigskip

Introduction to Category Theory with applications to Logic and Type Theory.

\emph{Not a spectator sport.}

Approach: Find an understanding somewhere between the concrete (examples) and
the abstract (``nonsense'').

Looking at \textbf{Universal Properties} through example problems.

\subsection{Adding an element to a set}

Given a set $S$, consider

\textbf{data:} $$ S \overset{f}{\rightarrow} S' \ni x $$ \\
\textbf{criteria:} $$
    \begin{tikzcd}
        S \arrow[r, "f_1"]
          \arrow[dr, swap, "f_2"]
          &
        S_1 \ni x_1 \arrow[d, "h"]
          \\
          {}&
        S_2 \in x_2
    \end{tikzcd}
$$

\begin{framed}
Diagrams such as these implicitly assume that compositions commute:
function $h : S_1 \rightarrow S_2$ must satisfy $h(x_1) = x_2$ and $\forall x
\in S .\; h(f_1(x)) = f_2(x)$.

So we can express the above using more concisely as $h \circ f_1 = f_2$.

$$
\begin{tikzcd}
    A \arrow[r, "f"] \arrow[rr, swap, bend right=40, "g \circ f"] & B \arrow[r, "g"] & C
\end{tikzcd}
$$

\end{framed}

\subsubsection*{Initial (Universal) Solution}
Initial: Any other solution can be mapped to from this solution. Note that the
solution doesn't always exist.

Find $S \overset{f}{\rightarrow} T \ni t$ such that
$$
\forall S \overset{f'}{\rightarrow} S' \ni x \ .\ 
\exists!\ h : T \rightarrow S' \text{ where }
h(t) = x \text{ and }
h \circ f = f'
$$

Solution

$$
    \begin{tikzcd}
        S \arrow[r, "\iota"]
          \arrow[dr, swap, "f"]
          & 
        S \cupplus \{\ast\} \ni \ast
          \arrow[d, "h"]
          \\
        {}
          &
        T \ni t
    \end{tikzcd}
$$

Where the disjoint union $\cupplus$ is defined as $\{\ast\} \cup \{ \lbrack
x\rbrack | x \in S \}$, and the square brackets tag the elements from $S$.

And defining $h$ as
\begin{equation*}
    h(z) = \begin{cases}
        f(x) & \text{if } z = \lbrack x \rbrack \\
        t & \text{if } z = \ast
    \end{cases}
\end{equation*}

For universality we must also prove that this is a unique solution.
Consider a $k$ that satisfies $k \circ \iota = f$, and $k(\ast) = t$. Then RTP
$k(z) = h(z) \forall z \in S \cupplus \{ \ast \}$.

\subsubsection*{Final solution}
There is also a final solution. Any other solution can map to the final
solution.

$$
    \begin{tikzcd}
        S \arrow[r, "f"]
          \arrow[dr, swap, "\forall g"]
          & 
        T \ni t
          \\
        {}
          &
        \mathcal{U} \ni u
          \arrow[u, swap, "h"]
    \end{tikzcd}
$$

Let $!_A$ be the constantly $\ast$ function, $!_A : A \rightarrow \{ \ast \}$.

So $f = !_S$, $h = !_{\mathcal{U}}$, $T = \{ \ast \}$, and $t = \ast$.

Note the correspondence between $\ast \in \{ \ast \}$ and \texttt{(): unit} in
most functional programming languages.

\subsection{Adding two sets}

Given $A$ and $B$ sets.

\textbf{data:}
$$
\begin{tikzcd}
    A \arrow[r, "f"] & C & B \arrow[l, swap, "g"]
\end{tikzcd}
$$

\textbf{criteria:}
$$
\begin{tikzcd}
    {} & C_1 \arrow[dd, "h"] & {} \\
    A \arrow[ru, "f_1"] \arrow[rd, swap, "f_2"]
    & {}
    &
    B \arrow[lu, swap, "g_1"] \arrow[ld, "g_2"]
    \\
    {} & C_2 & {}
\end{tikzcd}
$$

Such that (1) $h\ :\ C_1 \rightarrow C_2$ and (2) it commutes (i.e. $h \circ f_1 = f_2$
and $h \circ g_1 = g_2$).

\begin{exercise}
    Find the initial solution.
\end{exercise}

\subsection{Duality}

\subsubsection*{Duality}
When you have a problem in one direction you can turn it around and get a
problem in the other direction.

\subsubsection*{Dual form of adding two sets}

Given two sets $A$ and $B$.

\textbf{data:}
$$
\begin{tikzcd}
    A & C \arrow[l, swap, "f"] \arrow[r, "g"] & B
\end{tikzcd}
$$

\textbf{criteria:}
$$
\begin{tikzcd}
    {} & C_1 \arrow[dd, "h"] \arrow[ld, swap, "f_1"] \arrow[rd, "g_1"] & {} \\
    A & {} & B \\
    {} & C_2 \arrow[lu, "f_2"] \arrow[ru, swap, "g_2"] & {}
\end{tikzcd}
$$

Such that (1) $h\ :\ C_1 \rightarrow C_2$ and (2) it commutes (i.e. $h \circ f_2 = f_1$
and $h \circ g_2 = g_1$).

\begin{exercise}
    Find the final solution.
\end{exercise}

When considering the dual problem, the final solution is the one that is
interesting.

\subsection{Generating a monoid}

\subsubsection*{Monoids}

A monoid is a triple $(M, e, \ast)$ where $M$ is a set, $e \in M$ is a neutral
element, and $\ast$ is a binary operation. Where $e$ and $\ast$ obey:
\begin{align*}
    \forall x \in M\ .\ e \ast x = x \text{ and } x \ast e = x && \text{(neutral
    element)} \\
    (x \ast y) \ast z = x \ast (y \ast z) && \text{(associativity)}
\end{align*}

\subsubsection*{Monoid homomorphism}
A monoid homomorphism is a function between two monoids that preserves the
monoid structure. Formally, $h : (M_1, e_1, \ast_1)
\rightarrow (M_2, e_2, \ast_2)$ such that $h(e_1) = e_2$ and $\forall x, y \in
M_1\ .\ h(x \ast_1 y) = h(x) \ast_2 h(y)$.

Given a set $S$, and where $M, M_1, M_2$ are monoids.

\textbf{data:}
$$S \overset{f_1}{\rightarrow}M_1$$

\textbf{criteria:}
$$
\begin{tikzcd}
    S \arrow[r, "f_1"] \arrow[dr, swap, "f_2"] & M_1 \arrow[d, "h"] \\
    {} & M_2
\end{tikzcd}
$$

where $h$ is a monoid homomorphism.

\begin{exercise}
    Solve for the initial.
\end{exercise}

\end{document}
