\documentclass[a4paper, 12pt]{article}
\usepackage{a4wide}

\title{L108: Category Theory, Type Theory, and Logic\\
  ACS Category Theory Notes 2017}
\usepackage{nth}
\date{\nth{5} October 2017}
\author{Marcelo Fiore\\ Dhruv Makwana\\ Shaun Steenkamp}

\usepackage{fourier} % Better font
\usepackage{inconsolata} % mono spaced font
\usepackage{amssymb,amsthm,amsmath}
\usepackage{fdsymbol}
\usepackage{framed}
\usepackage{tikz}
\usepackage{tikz-cd}
\usepackage{microtype}

\theoremstyle{definition}
\newtheorem{definition}{Definition}
\newtheorem{lemma}{Lemma}
\newtheorem{theorem}{Theorem}
\newtheorem{exercise}{Exercise}

%%%% Stylistic
\setlength\parindent{0pt}
\setlength\parskip{0.5em}

%%%% Category Theory Macros
\newcommand {\cat}{%
    \mathbf%
}
\newcommand {\domain}[1] {%
    \mathrm{dom}(#1)%
}
\newcommand {\codomain}[1] {%
    \mathrm{cod}(#1)%
}
\newcommand {\idarrow}[1][] {%
    \mathbf{1}{#1}%
}
%%%% Macros for specific categories
\newcommand {\Cat}{%
    \cat {Cat}%
}
\newcommand {\Mon}{%
    \cat {Mon}%
}
\newcommand {\Poset}{%
    \cat {Poset}%
}
\newcommand {\Rel}{%
    \cat {Rel}%
}
\newcommand {\Sets}{%
    \cat {Sets}%
}
\newcommand {\Groups}{%
    \cat {Groups}%
}
\newcommand {\Graphs}{%
    \cat {Graphs}%
}

\newcommand{\ie}{\emph{i.e.}}

\newcommand{\eqdef}{\stackrel{\text{def}}{=}} % composition
\newcommand{\comp}{\circ} % composition
\newcommand{\icomp}{\,} % implicit composition

\newcommand{\setof}[1]{ \{ #1 \} } 
\newcommand{\bigsetof}[1]{ \big\{ #1 \big\} } 
\newcommand{\suchthat}{\mid}
\newcommand{\union}{\cup}

\begin{document}

\maketitle

\tableofcontents

\newpage
\section{Lecture 1 (\nth{5} October): How to think Categorically}
\vspace*{-5mm}\hspace*{7.75mm}
Marcelo Fiore, Dhruv Makwana, Shaun Steenkamp
\bigskip

Introduction to Category Theory with applications to Logic and Type Theory.

\emph{Not a spectator sport.}

Approach: Find an understanding somewhere between the concrete (examples) and
the abstract (``nonsense'').

Looking at \textbf{Universal Properties} through example problems.

\subsection{Adding an element to a set}

Given a set $S$, consider

\textbf{data:} $$ S \overset{f}{\rightarrow} S' \ni x $$ \\
\textbf{criteria:} $$
    \begin{tikzcd}
        S \arrow[r, "f_1"]
          \arrow[dr, swap, "f_2"]
          &
        S_1 \arrow[d, "h"] & \ni & x_1 \arrow[d, mapsto]
          \\
          {}&
        S_2 & \ni & x_2
    \end{tikzcd}
$$
\quad function $h : S_1 \rightarrow S_2$ must satisfy $h(x_1) = x_2$ and 
$\forall x \in S .\; h(f_1(x)) = f_2(x)$.

\begin{framed}
Diagrams such as the above implicitly assume that compositions commute.  We
can express the above using more concisely as:
\begin{center}
$h\comp f_1 = f_2$
\quad or \quad
$h\icomp f_1 = f_2$
\end{center}
where 
$$
\begin{tikzcd}
    A \arrow[r, "f"] \arrow[rr, swap, bend right=40, "g \circ f"] & B \arrow[r, "g"] & C
\end{tikzcd}
$$
with $(g\comp f)(a)\eqdef g(f(a))$ for all $a\in A$.
\end{framed}

\subsubsection*{Initial (Universal) Solution}

\emph{Initial}: Any other solution can be mapped to from this solution. 

Note that the solution doesn't always exist.

Find $S \overset{f}{\rightarrow} T \ni t$ such that
$$
\forall S \overset{f'}{\rightarrow} S' \ni x \ .\ 
\exists!\ h : T \rightarrow S' \text{ such that }
h(t) = x \text{ and }
h \circ f = f'
$$

Solution
$$
    \begin{tikzcd}
        S \arrow[r, "\iota"]
          \arrow[dr, swap, "f"]
          & 
        S_{\ast} 
        \arrow[d, "h"]
        & \ni & \ast
        \arrow[d, mapsto]
          \\
        {}
          &
        T & \ni & t
    \end{tikzcd}
$$
where $S_\ast$ is defined as 
$\setof{ \ast } \union \bigsetof{ \lbrack x\rbrack \suchthat x \in S }$
and $\iota: S \to S_\ast$ as $\iota(x)=[x]$ for all $x\in S$.  
%
(The set $S_\ast$ is an \emph{implementation} of the disjoint union of
$\setof{*}$ and $S$; the square brackets tag the elements from $S$.)

Define $h$, for $z\in S_\ast$, as
\begin{equation*}
    h(z) = \begin{cases}
        t & \text{if } z = \ast\\
        f(x) & \text{if } z = \lbrack x \rbrack 
    \end{cases}
\end{equation*}

For universality we must also prove that this is a unique solution.
Consider $k:S_\ast\to T$ that satisfies $k \circ \iota = f$, and 
$k(\ast) = t$. Then, we are RTP: $k(z) = h(z)\ \forall z \in S_\ast$.

\subsubsection*{Final solution}

There is also a final solution: any other solution can map to the final one.
$$
    \begin{tikzcd}
        S \arrow[r, "f"]
          \arrow[dr, swap, "\forall g"]
          & 
        T & \ni & t
          \\
        {}
          &
          U 
          \arrow[u, swap, "\exists! h"]
          & \ni & u
          \arrow[u, swap, mapsto]
    \end{tikzcd}
$$

For a set $A$, let $!_A$ be the unique function $A\to \setof{\ast}$ given by
the constantly $\ast$ function, that is, $!_A(a) = \ast$ for all $a\in A$.

Then, $T=\setof{\ast}$, $t=\ast$, and $f=\ !_S$ is a final solution.

(Note the correspondence between $\ast \in \{ \ast \}$ and \texttt{(): unit}
in most functional programming languages.)

\subsection{Adding two sets}

Given $A$ and $B$ sets.

\textbf{data:}
$$
\begin{tikzcd}
    A \arrow[r, "f"] & C & B \arrow[l, swap, "g"]
\end{tikzcd}
$$

\textbf{criteria:}
$$
\begin{tikzcd}
    {} & C_1 \arrow[dd, "h"] & {} \\
    A \arrow[ru, "f_1"] \arrow[rd, swap, "f_2"]
    & {}
    &
    B \arrow[lu, swap, "g_1"] \arrow[ld, "g_2"]
    \\
    {} & C_2 & {}
\end{tikzcd}
$$
That is, such that (1) $h\ :\ C_1 \rightarrow C_2$ and (2) it 
commutes~(\ie~$h \circ f_1 = f_2$ and $h \circ g_1 = g_2$).

\begin{exercise}
    Find the initial solution.
\end{exercise}

\subsection{Dual form}

\subsubsection*{Duality}

When you have a problem in ``one direction'' you can turn it around and get a
problem in ``the other direction''.

\subsubsection*{Dual form of adding two sets}

Given two sets $A$ and $B$.

\textbf{data:}
$$
\begin{tikzcd}
    A & C \arrow[l, swap, "f"] \arrow[r, "g"] & B
\end{tikzcd}
$$

\textbf{criteria:}
$$
\begin{tikzcd}
    {} & C_1 \arrow[dd, "h"] \arrow[ld, swap, "f_1"] \arrow[rd, "g_1"] & {} \\
    A & {} & B \\
    {} & C_2 \arrow[lu, "f_2"] \arrow[ru, swap, "g_2"] & {}
\end{tikzcd}
$$
That is, such that (1)~$h\ :\ C_1 \rightarrow C_2$ and (2)~it 
commutes~(\ie~$h \circ f_2 = f_1$ and $h \circ g_2 = g_1$).

\begin{exercise}
    Find the final solution.
\end{exercise}

Typically, when considering the dual problem, the dual solution is the one
that is of interest.

\subsection{Generating a monoid}

\subsubsection*{Monoids}

A monoid is a triple $(M, e, \ast)$ where $M$ is a set, $e \in M$ is a neutral
element, and $\ast$ is a binary operation on $M$; where $e$ and $\ast$ obey:
\begin{align*}
    & \forall x \in M\ .\ 
    e \ast x = x \text{ and } x \ast e = x && \text{(neutral
    element)} \\
    & \forall x,y,z \in M\ .\ 
    (x \ast y) \ast z = x \ast (y \ast z) && \text{(associativity)}
\end{align*}

\subsubsection*{Monoid homomorphism}

A monoid homomorphism is a function between two monoids that preserves the
structure. Formally, $h : (M_1, e_1, \ast_1) \rightarrow (M_2, e_2, \ast_2)$
is a function $h: M_1\to M_2$ such that $h(e_1) = e_2$ and $\forall x, y \in
M_1\ .\ h(x \ast_1 y) = h(x) \ast_2 h(y)$.

Given a set $S$, consider

\textbf{data:}
$$
S \overset{f_1}{\longrightarrow}M \qquad \text{$(M,e,\ast)$ a monoid}
$$

\textbf{criteria:}
$$
\begin{tikzcd}
    S \arrow[r, "f_1"] \arrow[dr, swap, "f_2"] & 
    M_1 
    \arrow[d, "h"] 
    & (M_1,e_1,\ast_1) 
    \arrow[d, "\text{$h$ an homomorphism}"] 
    & \text{a monoid}
    \\
    {}  
    & M_2 & (M_2,e_2,\ast_2) & \text{a monoid}
\end{tikzcd}
$$

\begin{exercise}
    Solve for the initial.
\end{exercise}

\end{document}
