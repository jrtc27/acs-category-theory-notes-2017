%%%%%%%%%%%%%%%%%%%%%%%% NOTES TO CONTRIBUTORS %%%%%%%%%%%%%%%%%%%%%%%%%%%%%%%%%
% - try to stick to a minimal package set unless you need them. In other words,
% don't use packages you don't understand simply because you're used to using
% them.
% - $$ eqn $$ is less universal than \[ eqn \] or \begin{equation} eqn \end{eqn}
% - Don't use the math environment just to center things. That's what the center
% environment is for.
% - If you have other stylistic guidelines you want other poeple to keep in
% mind, put them here


% In the future we may want to break this into many subfiles (see the subfiles
% package) so that we're not all working on a single monolithic document that
% takes a long time to compile.

\documentclass[a4paper, 12pt, oneside]{book}

\usepackage{a4wide}
\usepackage{nth}
% The "better font" is not scalable, also less standard
%\usepackage{fourier} % Better font
%\usepackage{inconsolata} % mono spaced font
\usepackage{amssymb,amsthm,amsmath}
\usepackage{fdsymbol}
\usepackage{framed}
\usepackage{tikz}
\usepackage{tikz-cd}
\usepackage{microtype}

\usepackage{multicol}
\usepackage{enumitem}

\usepackage[utf8]{inputenc}

% Don't put formatting in the Title; this will show up in other places once 
% we add headers. Instead, do the formatting manually

\title{ACS L108 Lecture Notes: Category Theory, Type Theory, and Logic}
\author{Marcelo Fiore}
\date{Michaelmas Term 2017--18}

%%%% Environment for Labeling Chapters
\newcommand{\lecturedetails}[2]{
    \vspace*{-10mm}\hspace*{7.75mm}
    \parbox{0.7\textwidth} {
        #2\\
        \textit{#1} 
    }
    \vspace{10mm}
}

% make sure that chapters are called lectures
\makeatletter
\renewcommand{\@chapapp}{Lecture}
\makeatother

%%%%%

\theoremstyle{definition}
\newtheorem{definition}{Definition}
\newtheorem{lemma}{Lemma}
\newtheorem{theorem}{Theorem}
\newtheorem{exercise}{Exercise}

%%%% Stylistic
\setlength\parindent{0pt}
\setlength\parskip{0.7em}

%%%% Category Theory Macros
\newcommand {\cat}{%
    \mathbf%
}
\newcommand {\domain}[1] {%
    \mathrm{dom}(#1)%
}
\newcommand {\codomain}[1] {%
    \mathrm{cod}(#1)%
}
\newcommand {\idarrow}[1][] {%
    \mathbf{1}{#1}%
}
%%%% Macros for specific categories
\newcommand {\Cat}{%
    \cat {Cat}%
}
\newcommand {\Mon}{%
    \cat {Mon}%
}
\newcommand {\Poset}{%
    \cat {Poset}%
}
\newcommand {\Rel}{%
    \cat {Rel}%
}
\newcommand {\Sets}{%
    \cat {Sets}%
}
\newcommand {\Groups}{%
    \cat {Groups}%
}
\newcommand {\Graphs}{%
    \cat {Graphs}%
}

\newcommand{\ie}{\emph{i.e.}}

\newcommand{\eqdef}{\stackrel{\text{def}}{=}} % composition
\newcommand{\comp}{\circ} % composition
\newcommand{\icomp}{\,} % implicit composition

\newcommand{\setof}[1]{ \{ #1 \} }
\newcommand{\bigsetof}[1]{ \big\{ #1 \big\} }
\newcommand{\suchthat}{\mid}
\newcommand{\union}{\cup}

\newcommand{\nelem}[1]{ \mathbf{ #1 } }
\newcommand{\id}[1]{ \mathrm{id}_{ #1 } }
\newcommand{\nats}{\mathbb{N}}

\begin{document}

\begin{center} {\huge \sc
Category Theory, Type Theory, and Logic\\
  Lecture Notes\\[4mm]}
  \Large Marcelo Fiore
\end{center}


\begin{multicols}{3}[\section*{Contributors}]
Dhruv Makwana\\ 
Shaun Steenkamp\\
Andrej Ivašković\\
Oliver Richardson
\end{multicols}
\clearpage


\tableofcontents

\newpage
\chapter{How to think Categorically}
\lecturedetails{5 October 2017}{M Fiore, D Makwana, S Steenkamp, O Richardson}

Introduction to Category Theory with applications to Logic and Type Theory.

\emph{Not a spectator sport.}

The approach is to find an understanding somewhere between the concrete
(examples) and the abstract (``nonsense''). The heart of category theory is
universal properties, namely `initial', and `final' categories. Therefore,
rather than giving a definition of a category straight away, we will first
explore \emph{Universal Properties} through example problems; this will
constitute the first few lectures.

\section{Examples}
We will begin with well known mathematical operations, viewed from a universal,
categorical point of view.

\subsection{Adding an element to a set}
\label{add-element-set}

Given a set $S$, consider

\begin{align*}
    \textbf{data:}&\qquad S \overset{f}{\rightarrow} S^\prime \ni x \\
    \textbf{criteria:}& \qquad 
    \begin{tikzcd}[ampersand replacement=\&]
        S \arrow[r, "f_1"]
          \arrow[dr, swap, "f_2"]
          \&
        S_1 \arrow[d, "h"] \& \ni \& x_1 \arrow[d, mapsto]
          \\
          {}\&
        S_2 \& \ni \& x_2
    \end{tikzcd}
\end{align*}

The function $h : S_1 \rightarrow S_2$ must satisfy $h(x_1) = x_2$ and 
$\forall x \in S .\; h(f_1(x)) = f_2(x)$.

\begin{framed}
Diagrams such as the above implicitly assume that compositions commute (that's
why they're called commutative diagrams). We can express the above more
concisely as:
\begin{center}
$h\comp f_1 = f_2$
\quad or \quad
$h\icomp f_1 = f_2$
\end{center}
where 
$$
\begin{tikzcd}
    A \arrow[r, "f"] \arrow[rr, swap, bend right=40, "g \circ f"] & B \arrow[r, "g"] & C
\end{tikzcd}
$$
with $(g\comp f)(a)\eqdef g(f(a))$ for all $a\in A$.
\end{framed}

\subsubsection*{Initial (Universal) Solution}

An \emph{initial} solution is one that is a suitable domain to map to any other
solution. Note that such a solution doesn't always exist. In the case of adding
a single element to a set, the problem can be specified in terms of finding an
appropriate $S \overset{f}{\rightarrow} T \ni t$ such that

\begin{equation*}
  \forall S \overset{f'}{\rightarrow} S' \ni x \ .\ 
  \exists!\ h : T \rightarrow S' \text{ such that }
  h(t) = x \text{ and }
  h \circ f = f'
\end{equation*}

We now diagram our solution:
\begin{center}
    \begin{tikzcd}
        S \arrow[r, "\iota"]
          \arrow[dr, swap, "f"]
          & 
        S_{\ast} 
        \arrow[d, "h"]
        & \ni & \ast
        \arrow[d, mapsto]
          \\
        {}
          &
        T & \ni & t
    \end{tikzcd}
\end{center}

where $S_\ast$ is defined as 
$\setof{ \ast } \union \bigsetof{ \lbrack x\rbrack \suchthat x \in S }$
and $\iota: S \to S_\ast$ as $\iota(x)=[x]$ for all $x\in S$.  
%
(The set $S_\ast$ is an \emph{implementation} of the disjoint union of
$\setof{*}$ and $S$; the square brackets tag the elements from $S$.)

Define $h$, for $z\in S_\ast$, as
\begin{equation*}
    h(z) = \begin{cases}
        t & \text{if } z = \ast\\
        f(x) & \text{if } z = \lbrack x \rbrack 
    \end{cases}
\end{equation*}

For universality we must also prove that this is a unique solution.
Consider $k:S_\ast\to T$ that satisfies $k \circ \iota = f$, and 
$k(\ast) = t$. Then, we are required to prove (RTP):
$k(z) = h(z)\ \forall z \in S_\ast$.

\subsubsection*{Final solution}

There is a second kind of universal property, callled a final solution: any
other solution admits a map to the final one, again examining the case of adding
an element to a set, our criteria now is inverted. As we are looking for a final
solution, any $U$ must admit a map to our solution $T$, as shown below.
\begin{center}
    \begin{tikzcd}
        S \arrow[r, "f"]
          \arrow[dr, swap, "\forall g"]
          & 
        T & \ni & t
          \\
        {}
          &
          U 
          \arrow[u, swap, "\exists! h"]
          & \ni & u
          \arrow[u, swap, mapsto]
    \end{tikzcd}
\end{center}

For a set $A$, let $!_A$ be the unique function $A\to \setof{\ast}$ given by
the constantly $\ast$ function, that is, $!_A(a) = \ast$ for all $a\in A$.

Then, $T=\setof{\ast}$, $t=\ast$, and $f=\ !_S$ is a final solution.

(Note the correspondence between $\ast \in \{ \ast \}$ and \texttt{(): unit}
in most functional programming languages.)

\subsection{Adding two sets}

Given $A$ and $B$ sets.

\begin{align*}
    \textbf{data:} \qquad& \begin{tikzcd}[ampersand replacement=\&]
            A \arrow[r, "f"] \& C \& B \arrow[l, swap, "g"]
        \end{tikzcd}\\
    \textbf{criteria:} \qquad& \begin{tikzcd}[ampersand replacement=\&]
        {} \& C_1 \arrow[dd, "h"] \& {} \\
        A \arrow[ru, "f_1"] \arrow[rd, swap, "f_2"]
        \& {}
        \&
        B \arrow[lu, swap, "g_1"] \arrow[ld, "g_2"]
        \\
        {} \& C_2 \& {}
    \end{tikzcd}
\end{align*}

That is, such that (1) $h\ :\ C_1 \rightarrow C_2$ and (2) it 
commutes~(\ie~$h \circ f_1 = f_2$ and $h \circ g_1 = g_2$).

\begin{exercise}
    Find the initial solution. \label{ex:addsets}
\end{exercise}

\subsection{Duality}

When you have a problem in ``one direction'' you can turn it around and get a
problem in ``the other direction''. This involves flipping the directions of the
arrows in the commutative diagrams and inverting certain properties.

\subsubsection*{Example: Dual form of adding two sets}

Given two sets $A$ and $B$.

\begin{align*}
     \textbf{data:}\qquad & \begin{tikzcd}[ampersand replacement=\&]
            A \& C \arrow[l, swap, "f"] \arrow[r, "g"] \& B
        \end{tikzcd} \\
    \textbf{criteria:}\qquad & \begin{tikzcd}[ampersand replacement=\&]
          {} \& C_1 \arrow[dd, "h"] \arrow[ld, swap, "f_1"] \arrow[rd, "g_1"]
          \& {} \\ A \& {} \& B \\
          {} \& C_2 \arrow[lu, "f_2"] \arrow[ru, swap, "g_2"] \& {}
      \end{tikzcd}
\end{align*}



That is, such that (1)~$h\ :\ C_1 \rightarrow C_2$ and (2)~it 
commutes~(\ie~$h \circ f_2 = f_1$ and $h \circ g_2 = g_1$).

\begin{exercise}
    Find the final solution. \label{ex:dual}
\end{exercise}

It is worth noting that a dual solution can be easier to solve and allow insight
into the original problem, or vice versa.

\subsection{Generating a monoid}

\subsubsection*{Background on Monoids}

A \emph{monoid} is a triple $(M, e, \ast)$ where $M$ is a set, $e \in M$ is a
neutral element, and $\ast$ is a binary operation on $M$; where $e$ and $\ast$
obey:
\begin{align*}
    & \forall x \in M\ .\ 
    e \ast x = x \text{ and } x \ast e = x && \text{(neutral
    element)} \\
    & \forall x,y,z \in M\ .\ 
    (x \ast y) \ast z = x \ast (y \ast z) && \text{(associativity)}
\end{align*}

\subsubsection*{Monoid homomorphism}

A monoid homomorphism is a function between two monoids that preserves the
structure of the operation $\ast$. Formally,
$h : (M_1, e_1, \ast_1) \rightarrow (M_2, e_2, \ast_2)$ is a function
$h: M_1\to M_2$ such that $h(e_1) = e_2$ and
$\forall x, y \in M_1\ .\ h(x \ast_1 y) = h(x) \ast_2 h(y)$.

Our next exercise will deal with the univerally most general, or initial monoid.
Given a set $S$, we would like to construct  a monoid. In the language we've
been using, we can diagram it like this:

\begin{align*}
    \textbf{data:}\qquad& S \overset{f_1}{\longrightarrow}M
    \qquad \text{$(M,e,\ast)$ a monoid}\\
    \textbf{criteria:}\qquad& \begin{tikzcd}[ampersand replacement=\&]
        S \arrow[r, "f_1"] \arrow[dr, swap, "f_2"] \& 
        M_1 
        \arrow[d, "h"] 
        \& (M_1,e_1,\ast_1) 
        \arrow[d, "\text{$h$ an homomorphism}"] 
        \& \text{a monoid}
        \\
        {}  
        \& M_2 \& (M_2,e_2,\ast_2) \& \text{a monoid}
    \end{tikzcd}
\end{align*}

\begin{exercise}
    Find the initial solution in this case.
\end{exercise}

%%%%%%%%%%%%%%%%%%%%%%%%%%%%%%%%%%%%%%%%%%%%%%%%%%%%%%%%%%%%%%%%%%%%%%%%%%%%%%%%%%%%%%%%%%%%%%%%%%%%%%%%

\chapter{More universal problems}
\lecturedetails{10 October 2017}{M Fiore, A Ivašković, O Richardson }

The start of this lecture will provide some hints for the exercises from Lecture
1, and continue with a few more examples, aimed at continuing to build intuition.
This intuition is important because, again, keep in mind that there is no
general method for finding universal initial and final solutions.

\section{Hints To Exercises}

\subsection{Adding two sets}

Recall in Exercies \ref{ex:addsets}, we were looking for a universal initial
solution to adding sets $A$ and $B$:
\begin{center}
  \begin{tikzcd}
      {} & S \arrow[dd, dashed, "\exists! u"] & {} \\
      A \arrow[ru, "\alpha"] \arrow[rd, swap, "f"]
      & {}
      &
      B \arrow[lu, swap, "\beta"] \arrow[ld, "g"]
      \\
      {} & C & {}
  \end{tikzcd}
\end{center}
for all valid choices of $C$, $f$, and $g$.

When dealing with a problem we do not know immediately how to solve, it might be
useful to somehow `simplify' it in order to get some insight --- but not so much
that we lose all structural information.

Consider a special instance of this problem, where $B$ is a singleton set
$\setof{\ast}$:

\begin{center}
  \begin{tikzcd}
      {} & S & {} \\
      A \arrow[ru]
      & {}
      &
      \setof{\ast} \arrow[lu]
  \end{tikzcd}
\end{center}

But $\setof{\ast} \to S$ is mapping $\ast$ to some element in $S$, and overall
this represents the problem of adding an element to $A$
(subsection~\ref{add-element-set}). We want to make sure that there is a unique
mapping to any other solution $C$, as illustrated below.

\begin{center}
\begin{tikzcd}
    {} & S \arrow[dd, dashed, "\exists!"] & {} \\
    A \arrow[ru] \arrow[rd]
    & {}
    &
    \setof{\ast} \arrow[lu] \arrow[ld]
    \\
    {} & C & {}
\end{tikzcd}
\end{center}

Here $S$ has to preserve the structure of $A$ and be augmented with an
additional element in some way. The image of elements of $A$ in $S$ have to be
distinct from the image of $\ast$, since otherwise there would not be a unique
way of representing the two in $C$. This is a map $a \mapsto [a]$ for $a \in A$.
Compare this to the previous problem, where we considered
$A \to A_\ast = \setof{\ast} \union \setof{[x] \suchthat x \in A}$.

Taking this `tagging' idea further and generalising, we may consider the set:
\begin{equation*}
A \uplus B = \setof{(0, a) \suchthat a \in A} \union
\setof{(1, b) \suchthat b \in B}
\end{equation*}

We claim that $S$ should be $A \uplus B$, and that $\alpha: a \mapsto (0, a)$,
$\beta: b \mapsto (1, b)$ should be the maps for the initial solution.

\begin{exercise}
Show that this is indeed an initial solution.
\end{exercise}

\subsection{The dual problem}

In exercise \ref{ex:dual}, we were asked to find the dual solution to that
described previously:
\begin{center}
\begin{tikzcd}
    {} & S \arrow[ld] \arrow[rd] & {} \\
    A
    & {}
    &
    B
    \\
    {} & C \arrow[uu, dashed, "\exists!"] \arrow[lu, "f"] \arrow[ru, swap, "g"] & {}
\end{tikzcd}
\end{center}

This is a more involved problem than adding two sets, and simplifications might
not immediately reveal needs to be done. Even so, to give hints, we will test
different structures for $B$ to gain some intuition.

First, consider the one element set
$B = \setof{\ast} = \nelem{1}$\footnote{\textbf{Notation.} $\nelem{1}$ is a one
element set.} and then the problem becomes:
\begin{center}
\begin{tikzcd}
    {} & S \arrow[ld] \arrow[rd] & {} \\
    A
    & {}
    &
    \nelem{1}
    \\
    {} & X \arrow[uu, dashed] \arrow[lu] \arrow[ru] & {}
\end{tikzcd}
\end{center}

There is only one function $X \to \nelem{1}$ and only one function $S \to
\nelem{1}$ (they both map everything to $\ast$), so the problem is reduced to:
\begin{center}
\begin{tikzcd}
    {} & S \arrow[ld] \\
    A
    & {} \\
    {} & X \arrow[uu, dashed] \arrow[lu] 
\end{tikzcd}
\end{center}

and the final solution to this is clear:
$$
\begin{tikzcd}
    {} & A \arrow[ld, swap, "\id{A}"] \\
    A
    & {} \\
    {} & X \arrow[uu, dashed, "\exists! f"] \arrow[lu, "\forall f"]
\end{tikzcd}
$$

In general let's say $S$ is of the form $A \circledast B$ for some operation
$\circledast$, where for $B = \nelem{1}$ we have that
$A \circledast B \sim A$ \ie~the structure matches the structure of $A$ in some
way.

What if $B = \emptyset$?
$$
\begin{tikzcd}
    {} & A \circledast \emptyset \arrow[ld] \arrow[rd] & {} \\
    A
    & {}
    &
    \emptyset
    \\
    {} & X \arrow[uu, dashed] \arrow[lu] \arrow[ru] & {}
\end{tikzcd}
$$
This is only possible and valid when $A \circledast \emptyset = \emptyset$
because there are no functions from a nonempty set into an empty set. Also
recall that there is only one function from $\emptyset$ to any other set --
namely the empty function (the function with the empty graph).

And what if $B = \setof{0, 1} = \nelem{2}$ -- namely, what should
$A \circledast \nelem{2}$ be? This step gives a lot of insight about the final
solution.

\section{Isomorphism of initial solutions}

Recall that in our initial solution to the problem of generating a monoid
from a set, we actually found a number of different solutions, all of which
seemed somehow structurally similar:

\begin{enumerate}

\item $S \to \mathrm{List}(S)$, where $\mathrm{List}(S)$ is the monoid of lists
whose items have values drawn from $S$, with the empty list $[]$ being the
neutral element and the append operator $@$ being the underlying operation.

\item $S \to S^\ast$, where $S^\ast$ is the set of all strings/words on $S$.
This is a monoid if the neutral element is $\epsilon$ (the empty word) and the
operation is string concatenation $\cdot$.

\item For $n \in \nats$ define $S^n$ in the following way:
$$S^0 = \setof{(0, ())}$$
and for $n > 0$:
$$S^n = \setof{(n, s_1, \ldots, s_n) \suchthat s_1, \ldots, s_n \in S}$$

This is a monoid when the operation $\cdot$ behaves as:
$$(l, s_1, \ldots, s_l) \cdot (k, s_1^\prime, \ldots, s_k^\prime) =
(l + k, s_1, \ldots, s_l, s_1^\prime, \ldots, s_k^\prime)$$
with $(0, ())$ being the neutral element.

\end{enumerate}

Each of these solutions are indeed `essentially the same' and this is a
fundamental property of universal initial solutions. This concept (isomorphism)
will properly be defined once we start dealing with category theory, but for now
we can still prove that two initial solutions of the kind we've been dealing
with are isomorphic, \ie~that they are in bijection and share structure.

Given two initial solutions $M$ and $Y$, diagrammatically:
$$
\begin{tikzcd}
    S \arrow[r, "m"] \arrow[rd, swap, "y"]
    \arrow[rdd, swap, bend right = 20, "m"] & 
    M \arrow[d, dashed, "\exists! u"] \\
    {} & Y \arrow[d, dashed, "\exists! v"] \\
    {} & M
\end{tikzcd}
$$

Note that we have also:
$$
\begin{tikzcd}
    S \arrow[r, "m"] \arrow[rd, swap, "m"] & 
    M \arrow[d, dashed, "\exists! \id{M}"] \\
    {} & M
\end{tikzcd}
$$

Thus $v \comp u = \id{M}$. Analogously we conclude that $u \comp v = \id{Y}$
because:
$$
\begin{tikzcd}
    S \arrow[r, "y"] \arrow[rd, swap, "m"]
    \arrow[rdd, swap, bend right = 20, "y"] & 
    Y \arrow[d, dashed, "\exists! v"]
    \arrow[dd, dashed, bend left = 80, "\exists! \id{Y}"] \\
    {} & M \arrow[d, dashed, "\exists! u"] \\
    {} & Y
\end{tikzcd}
$$

From this we conclude that \textit{any two initial solutions are in bijective
correspondence}:

$$
\begin{tikzcd}
    M \arrow[r, bend left = 20] \arrow[loop left, "\id{M}"] & 
    Y \arrow[l, bend left = 20] \arrow[loop right, "\id{Y}"]
\end{tikzcd}
$$

One can thing of this from a computational perspective: the structural
properties are `implementation independent' -- structure should be preserved
regardless of the underlying implementation (it doesn't matter whether we're
dealing with strings, lists\ldots).

\begin{exercise}
From a set $S$ generate a \textit{commutative} monoid $M$ \ie~one that satisfies
$x \cdot y = y \cdot x$ for all $x, y \in M$ --- find the initial solution.
\end{exercise}

\begin{exercise}
From a set $S$ generate a commutative and \textit{idempotent} monoid $M$ \ie~one
that also satisfies $x \cdot x = x$ for all $x \in M$ --- find the initial
solution.
\end{exercise}

\section{Orders and lattices}

Let $P$ be a set, and $\leq\ \subseteq P \times P$ a binary relation. We will
start by defining a few nice properties for an order to have:

\begin{definition}
Reflexivity, transitivity, antisymmetry
\begin{itemize}[noitemsep,topsep=0pt]
    \item $\leq$ is \textit{reflexive} if $\forall a \in P.\quad a \leq a$
    \item $\leq$ is \textit{transitive} if
      $\forall a,b,c \in P. \quad a \leq b \wedge b \leq c \Rightarrow a \leq c$
    \item $\leq$ is \textit{antisymmetric} if $\forall a,b \in P. \quad
      a \leq b \wedge b \leq a \Rightarrow a = b$
\end{itemize}
\end{definition}

There is also special terminology for orders which obeys these axioms:

\begin{definition} Partial orders and preorders
\begin{itemize}[noitemsep,topsep=0pt]
\item $(P, \leq)$ is a \emph{preorder}, or \emph{quasiorder} if $\leq$ is
reflexive and transitive
\item $(P, \leq)$ is a \emph{partial order} if $\leq$ is a preorder (reflexive
and transitive) and $\leq$ is also antisymmetric
\end{itemize}
\end{definition}

We may define an algebraic structure on top of a partially ordered set:

\begin{definition}
A partial order $(P, \leq)$ augmented with an idempotent, commutative and
associative binary operation $\lor$\footnote{\textbf{join} or
    \textbf{disjunction}} that satisfies $x \leq y \iff x \vee y = x$ is a
\textbf{join semilattice}.

It may also have a least (bottom) element $\bot \in L$ satisfying
$\forall x \in L,~\bot\lor x = \bot$.
\end{definition}

\subsection{Exercises}

We can now diagram what we desire from an initial solution to a structure of
this kind. If $S$ is a set, then we want to construct an initial join
semilatice, as follows:
% Create an example environment?
\begin{align*}
    \textbf{data:}\qquad& S \overset{f_1}{\longrightarrow}L \qquad\qquad
    \text{$(L,\lor)$ a join semilatice}\\
    \textbf{criteria:}\qquad& \begin{tikzcd}[ampersand replacement=\&]
        S \arrow[r, "f_1"] \arrow[dr, swap, "f_2"] \& 
        \tilde{S}
        \arrow[d, "\exists ! h"] 
        \& (\tilde{S},\lor_{\tilde{S}},) 
        \arrow[d, "\text{$h$ a homomorphism}"] 
        \& \text{a universal (initial) join semilatice}
        \\
        {}  
        \& L \& (L,\lor_L) \& \text{any join semilatice}
    \end{tikzcd}\\
    &\text{where $h$ must preserve structure, \ie~$h(x\lor_{\tilde{S}} y) = h(x)
    \lor_L h(y)$}
\end{align*}
The exercise is to find the initial solution here, but we will start with some
intuition. The join semilatice generated from $S$ must contain each element
$x\in S$. Similarly, for any two, or three elements in $S$, it must contain the
disjunction of all of them. Since the operator is commutative, the order doesn't
matter, and since it's idempotent, only one copy of each element of $S$ can
exist. The picture looks something like this:
\begin{align*}
x  &\qquad\leftrightarrow\qquad \setof{x} \subseteq S \\
x \lor y &\qquad\leftrightarrow\qquad \setof{x,y} \subseteq S \\
x \lor y \lor z &\qquad\leftrightarrow\qquad \setof {x,y,z} \subseteq S \\
 &\qquad\vdots \\
x_1 \lor \cdots \lor x_n &\qquad\leftrightarrow\qquad \setof{x_1, \cdots x_n}
\subseteq S
\end{align*}

Thus, we might guess that the universal initial solution $\tilde{S}$ is the
collection of finite subsets of $S$:
$\mathcal{P}_{\mathrm{fin}}(S) = \setof{X \suchthat
X \subseteq_\mathrm{fin} S}$, with the join operator $X \lor Y = X \cup Y$.
This is close, but not entirely correct.

\begin{exercise}
There is a `bug' in this solution. Find the actual initial solution.
\end{exercise}

\end{document}
